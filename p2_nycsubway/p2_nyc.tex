\documentclass{article}
\usepackage[a4paper, portrait, margin=1.5cm]{geometry}
\usepackage[utf8]{inputenc}
\usepackage[english]{babel}
\usepackage{longtable}
\usepackage{graphicx}

\title{Data Exploration of NYC Subway Dataset}
\author{Ivailo Kassamakov}
\date{October 2015}
\begin{document}

\maketitle
\begin{abstract}
The present project is an effort towards fulfilling the requirements of the Data Analyst Nanodegree at Udacity.

The goal is to perform exploratory data analysis on a dataset collected from readings of weather sensors and turnstile counters situated at the stations of the NYC subway.
\end{abstract}

\section{Introduction}

The input dataset comes as a CSV file containing 42649 records. Each record consist of 27 columns, and their meaning is given in table \ref{tab:col_meaning}

\begin{longtable}[h]{c|l|p{8cm}}

\hline
Num & Column name & Column Description \\[3pt]
\hline
\endfirsthead

%\hline
\multicolumn{3}{c}{(Table Cont'd)}\\
\hline
Num & Column name & Column Description \\[3pt]
\hline

\endhead

\hline
\multicolumn{3}{c}{(Continues on next page)}\\
\endfoot


\hline
\caption{Meaning of the columns in the input dataset
\label{tab:col_meaning}}
\endlastfoot

 1 & UNIT & Remote unit that collects turnstile information. Can collect from multiple banks of turnstiles. Large subway stations can have more than one unit. \\[3pt]
 
2 & DATEn & Date in “yyyy-mm-dd” (2011-05-21) format. \\[3pt]

3 & TIMEn & Time in “hh:mm:ss” (08:05:02) format. \\[3pt]

4 & ENTRIESn & Raw reading of cumulative turnstile entries from the remote unit. Occasionally resets to 0. \\[3pt]

5 & EXITSn & Raw reading of cumulative turnstile exits from the remote unit. Occasionally resets to 0. \\[3pt]

6 & ENTRIESn\_hourly & Difference in ENTRIES from the previous REGULAR reading.\\[3pt] 

7 & EXITSn\_hourly & Difference in EXITS from the previous REGULAR reading. \\[3pt]

8 & datetime & Date and time in “yyyy-mm-dd hh:mm:ss” format (2011-05-01 00:00:00). Can be parsed into a Pandas \tt datetime \rm object without modifications. \\[3pt]

9 & hour & Hour of the timestamp from TIMEn. Truncated rather than rounded. \\[3pt]

10 & day\_week & Integer (0--6 Mon--Sun) corresponding to the day of the week. \\[3pt]

11 & weekday & Indicator (0 or 1) if the date is a weekday (Mon--Fri). \\[3pt]

12 & station & Subway station corresponding to the remote unit. \\[3pt]

13 & latitude & Latitude of the subway station corresponding to the remote unit.  \\[3pt]

14 & longitude & Longitude of the subway station corresponding to the remote unit. \\[3pt]

15 & conds & Categorical variable of the weather conditions (Clear, Cloudy etc.) for the time and location. \\[3pt]

16 & fog & Indicator (0 or 1) if there was fog at the time and location. \\[3pt]

17 & precipi & Precipitation in inches at the time and location. \\[3pt]

18 & pressurei & Barometric pressure in inches Hg at the time and location. \\[3pt] 

19 & rain  & Indicator (0 or 1) if rain occurred within the calendar day at the location. \\[3pt]

20 & tempi & Temperature in $^{\circ}{\rm F}$ at the time and location. \\[3pt]

21 & wspdi & Wind speed  in mph at the time and location. \\[3pt]

22 & meanprecipi & Daily average of \tt precipi \rm for the location. \\[3pt]

23 & meanpressurei & Daily average of \tt pressurei \rm for the location. \\[3pt]

24 & meantempi & Daily average of \tt tempi \rm for the location. \\[3pt]

25 & meanwspdi & Daily average of \tt wspdi \rm for the location. \\[3pt]

26 & weather\_lat & Latitude of the weather station the weather data is from. \\[3pt]

27 & weather\_lon & Longitude of the weather station the weather data is from. 
 
\end{longtable}

The following exploratory and graphic analysis will be performed:
\begin{itemize}
\item Map the subway stations.
\item Order the subway stations by number of UNITS in them.
\item Find out if people ride more during rainy days.
\item Determine ridership as a function of time-of-day.
\item Determine ridership as a function of day-of-week.
\item Create a predictive model for the ridership.


\end{itemize}

\section{Some information about the NYC subway}

Looking at the input dataset we can glean the following information about the NYC subway:
\begin{itemize}
\item There are 207 subway stations.
\item There are 240 UNITs, indicating that some stations have more than one unit (\emph{multi-unit} stations).
\item There are 28 multi-unit stations, as given in table \ref{tab:multi-unit-stations}
\item The dataset has been collected during the period 5/1/11-5/31/11, at times 00:00:00, 04:00:00, 08:00:00, 12:00:00, 16:00:00, 20:00:00. For most, but not all UNITs data has been collected for each of the 31 days of the period. For the days, when data has been collected, this has been done for all 6 time points. The UNITs  which miss data for some days of the period are: R228, R253, R260, R273, R295, R356, R453, R459.
\end{itemize}

\begin{table}[h]
\centering
\begin{tabular}{l|c}
\hline
Station & Number of UNITs\\
\hline
111 ST &            2\\[3pt]
125 ST  &           2\\[3pt]
145 ST   &          3\\[3pt]
167 ST    &         2\\[3pt]
18 AVE     &        2\\[3pt]
23 ST-6 AVE &       2\\[3pt]
25 ST        &      2\\[3pt]
34 ST-HERALD SQ &   2\\[3pt]
34 ST-PENN STA  &   3\\[3pt]
42 ST-TIMES SQ  &   2\\[3pt]
50 ST           &   3\\[3pt]
86 ST           &   3\\[3pt]
ATLANTIC AVE    &   2\\[3pt]
BOWLING GREEN   &   2\\[3pt]
CHAMBERS ST     &   2\\[3pt]
CHURCH AVE      &   2\\[3pt]
DEKALB AVE      &   2\\[3pt]
FORDHAM ROAD    &   2\\[3pt]
GRAND ST        &   2\\[3pt]
HOYT ST         &   2\\[3pt]
JAY ST-METROTEC &   2\\[3pt]
LEXINGTON AVE   &   3\\[3pt]
LEXINGTON-53 ST &   2\\[3pt]
NOSTRAND AVE    &   2\\[3pt]
PROSPECT AVE    &   2\\[3pt]
RECTOR ST       &   2\\[3pt]
SPRING ST       &   2\\[3pt]
WALL ST         &   2\\
\hline
\end{tabular}
\caption{Multi-unit subway stations}
\label{tab:multi-unit-stations}
\end{table}


\section {Input dataset correctness}

To judge the quality of the input dataset, we will check it for missing values in the columns, as well as evaluate the range of data in each column.

A quick check with \tt df.apply(pd.isnull,axis=0).any() \rm for missing/NaN values shows that all columns of the input dataset contain \textbf{valid} values. 

The \tt describe() \rm function of the input DataFrame returns the data ranges shown in Table \ref{tab:input-set-describe}. A simple visual check confirms that the data are in their expected ranges, and there are not outliers.

% Please add the following required packages to your document preamble:
% \usepackage{graphicx}
\begin{table}[]
\centering
%\resizebox{\textwidth}{!}{%
\begin{tabular}{l| llll}
\hline
Description      & Count &  min & max           \\
\hline
ENTRIESn         & 42649 &  0   & 23577460  \\
EXITSn           & 42649 &  0   & 14937820  \\
ENTRIESn\_hourly & 42649 &  0   & 32814  \\
EXITSn\_hourly   & 42649 &  0   & 34828  \\
hour             & 42649 &  0   & 20  \\
day\_week        & 42649 &  0   & 6  \\
weekday          & 42649 &  0   & 1  \\
latitude         & 42649 &  40.576152  & 40.88918  \\
longitude        & 42649 &  -74.073622 & -73.75538 \\
fog              & 42649 &  0   & 1  \\
precipi          & 42649 &  0   & 0.3  \\
pressurei        & 42649 &  29.55  & 30.32  \\
rain             & 42649 &  0   & 1  \\
tempi            & 42649 &  46.90  & 86.00  \\
wspdi            & 42649 &  0   & 23.00  \\
meanprecipi      & 42649 &  0   & 0.1575  \\
meanpressurei    & 42649 &  29.59 & 30.29333  \\
meantempi        & 42649 &  49.40  & 79.80  \\
meanwspdi        & 42649 &  0   & 17.08333  \\
weather\_lat     & 42649 &  40.600204  & 40.86206  \\
weather\_lon     & 42649 &  -74.014870 & -73.69418 \\
\hline
\end{tabular}
%}
\caption{Description of the input dataset}
\label{tab:input-set-describe}
\end{table}

\section{Visualizing of the subway stations}

Figure \ref{fig:subway_stations} shows a map of the subway UNITs, according to the geographical coordinates contained in the input set.

\begin{figure}[h]
\centering
\includegraphics[width=\textwidth]{map_stations.png}
\caption{Map of NYC subway stations}
\label{fig:subway_stations}
\end{figure}



\end{document}
