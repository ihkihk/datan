\documentclass{article}
\usepackage[utf8]{inputenc}
\usepackage[english]{babel}
\usepackage{longtable}

\title{Data Exploration of NYC Subway Dataset}
\author{Ivailo Kassamakov}
\date{October 2015}
\begin{document}

\maketitle
\begin{abstract}
The present project is an effort towards fulfilling the requirements of the Data Analyst Nanodegree at Udacity.

The goal is to perform exploratory data analysis on a dataset collected from readings of weather sensors and turnstile counters situated at the stations of the NYC subway.
\end{abstract}

\section{Introduction}

The input dataset comes as a CSV file containing 42649 records. Each record consist of 27 columns, and their meaning is given in table \ref{tab:col_meaning}

\begin{longtable}[h]{c|l|p{8cm}}

\hline
Num & Column name & Column Description \\[3pt]
\hline
\endfirsthead

%\hline
\multicolumn{3}{c}{(Table Cont\'d)}\\
\hline
Num & Column name & Column Description \\[3pt]
\hline

\endhead

\hline
\multicolumn{3}{c}{(Continues on next page)}\\
\endfoot


\hline
\caption{Meaning of the columns in the input dataset
\label{tab:col_meaning}}
\endlastfoot

 1 & UNIT & Remote unit that collects turnstile information. Can collect from multiple banks of turnstiles. Large subway stations can have more than one unit. \\[3pt]
 
2 & DATEn & Date in “yyyy-mm-dd” (2011-05-21) format. \\[3pt]

3 & TIMEn & Time in “hh:mm:ss” (08:05:02) format. \\[3pt]

4 & ENTRIESn & Raw reading of cumulative turnstile entries from the remote unit. Occasionally resets to 0. \\[3pt]

5 & EXITSn & Raw reading of cumulative turnstile exits from the remote unit. Occasionally resets to 0. \\[3pt]

6 & ENTRIESn\_hourly & Difference in ENTRIES from the previous REGULAR reading.\\[3pt] 

7 & EXITSn\_hourly & Difference in EXITS from the previous REGULAR reading. \\[3pt]

8 & datetime & Date and time in “yyyy-mm-dd hh:mm:ss” format (2011-05-01 00:00:00). Can be parsed into a Pandas \tt datetime \rm object without modifications. \\[3pt]

9 & hour & Hour of the timestamp from TIMEn. Truncated rather than rounded. \\[3pt]

10 & day\_week & Integer (0--6 Mon--Sun) corresponding to the day of the week. \\[3pt]

11 & weekday & Indicator (0 or 1) if the date is a weekday (Mon--Fri). \\[3pt]

12 & station & Subway station corresponding to the remote unit. \\[3pt]

13 & latitude & Latitude of the subway station corresponding to the remote unit.  \\[3pt]

14 & longitude & Longitude of the subway station corresponding to the remote unit. \\[3pt]

15 & conds & Categorical variable of the weather conditions (Clear, Cloudy etc.) for the time and location. \\[3pt]

16 & fog & Indicator (0 or 1) if there was fog at the time and location. \\[3pt]

17 & precipi & Precipitation in inches at the time and location. \\[3pt]

18 & pressurei & Barometric pressure in inches Hg at the time and location. \\[3pt] 

19 & rain  & Indicator (0 or 1) if rain occurred within the calendar day at the location. \\[3pt]

20 & tempi & Temperature in $^{\circ}{\rm F}$ at the time and location. \\[3pt]

21 & wspdi & Wind speed  in mph at the time and location. \\[3pt]

22 & meanprecipi & Daily average of \tt precipi \rm for the location. \\[3pt]

23 & meanpressurei & Daily average of \tt pressurei \rm for the location. \\[3pt]

24 & meantempi & Daily average of \tt tempi \rm for the location. \\[3pt]

25 & meanwspdi & Daily average of \tt wspdi \rm for the location. \\[3pt]

26 & weather\_lat & Latitude of the weather station the weather data is from. \\[3pt]

27 & weather\_lon & Longitude of the weather station the weather data is from. 
 
\end{longtable}

The following exploratory and graphic analysis will be performed:
\begin{itemize}
\item{ Map the subway stations}
\item{ Order the subway stations by number of UNITS in them}
\item{ Find out if people ride more during rainy days}
\item Find out if

\end{itemize}

\section{Some information about the NYC subway}

Looking at the input dataset we can glean the following information about the NYC subway:
\begin{itemize}
\item There are 207 subway stations.
\item There are 240 UNITs, indicating that some stations have more than one unit.
\end{itemize}

\end{document}
